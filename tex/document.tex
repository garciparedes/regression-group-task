% !TEX root = ./document.tex

\documentclass{article}

\usepackage{mystyle}
\usepackage{myvars}

%-----------------------------

\begin{document}

  \maketitle

  %-----------------------------
  %  TEXT
  %-----------------------------

  \part{Ejercicio Kutner:\\ Mantenimiento de Copiadoras}

    \setcounter{section}{1}

    \setcounter{subsection}{19}
    \subsection{\textbf{Copier maintenance}. The Tri-City Office Equipment Corporation sells an imported copier on a franchise basis and performs preventive maintenance and repair service on this copier. The data below have been collected from 45 recent calls on users to perform routine preventive maintenance service; for each call, $X$ is the number of copiers serviced and $Y$ is the total number of minutes spent by the service person. Assume that first-order regression model (1.1) is appropriate.}
    \label{sec:e1-20}

      \subsubsection{Obtain the estimated regression function.}

        \paragraph{}
        [TODO ]

      \subsubsection{Plot the estimated regression function and the data. How well does the estimated regression function fit the data?}

        \paragraph{}
        [TODO ]

      \subsubsection{Interpret $\widehat{\beta_0}$ in your estimated regression function. Does bo provide any relevant information here? Explain.}

        \paragraph{}
        [TODO ]

      \subsubsection{Obtain a point estimate of the mean service time when $X = 5$ copiers are serviced.}

        \paragraph{}
        [TODO ]

    \setcounter{subsection}{23}
    \subsection{Refer to \textbf{Copier maintenance} Problem \ref{sec:e1-20}.}

      \paragraph{}
      [TODO ]

      \subsubsection{Obtain the residuals $e_i$ and the sum of the squared residuals $\sum_i e_i^2$. What is the relation between the sum of the squared residuals here and the quantity $Q$ in (1.8)?}

        \paragraph{}
        [TODO ]

      \subsubsection{Obtain point estimates of $\sigma^2$ and $\sigma$. In what units is $\sigma$ expressed?}

        \paragraph{}
        [TODO ]

    \setcounter{section}{2}

    \setcounter{subsection}{4}
    \subsection{Refer to \textbf{Copier maintenance} Problem \ref{sec:e1-20}.}

      \paragraph{}
      [TODO ]

      \subsubsection{Estimate the change in the mean service time when the number of copiers serviced increases by one. Use a $90\%$ confidence interval. Interpret your confidence interval.}

        \paragraph{}
        [TODO ]

      \subsubsection{Conduct a t test to determine whether or not there is a linear association between $X$ and $Y$ here; control the $\alpha$ risk at $0.10$. State the alternatives, decision rule, and conclusion. What is the \emph{P-value} of your test?}

        \paragraph{}
        [TODO ]

      \subsubsection{Are your results in parts (a) and (b) consistent? Explain.}

        \paragraph{}
        [TODO ]

      \subsubsection{The manufacturer has suggested that the mean required time should not increase by more
than $14$ minutes for each additional copier that is serviced on a service call. Conduct a test to decide whether this standard is being satisfied by Tri-City. Control the risk of a Type I error at $0.05$. State the alternatives, decision rule, and conclusion. What is the \emph{P-value} of the test?}

        \paragraph{}
        [TODO ]

      \subsubsection{Does $\widehat{\beta_0}$ give any relevant information here about the start-up time on calls-i.e., about the time required before service work is begun on the copiers at a customer location?}

        \paragraph{}
        [TODO ]

    \setcounter{subsection}{13}
    \subsection{Refer to \textbf{Copier maintenance} Problem \ref{sec:e1-20}.}

      \paragraph{}
      [TODO ]

      \subsubsection{Obtain a $90\%$ confidence interval for the mean gervice time on calls in which six copiers are serviced. Interpret your confidence interval.}

        \paragraph{}
        [TODO ]

      \subsubsection{Obtain a $90\%$ prediction interval for the service time on the next call in which six copiers are serviced. Is your prediction interval wider than the corresponding confidence interval in part (a)? Should it be?}

        \paragraph{}
        [TODO ]

    \setcounter{subsection}{23}
    \subsection{Refer to \textbf{Copier maintenance} Problem \ref{sec:e1-20}.}


      \paragraph{}
      [TODO ]

      \setcounter{subsection}{1}
      \subsubsection{Conduct an \emph{F-test} to determine whether or not there is a linear association between time spent and number of copiers serviced; use $\alpha = 0.10$. State the alternatives, decision rule, and conclusion.}

        \paragraph{}
        [TODO ]

      \subsubsection{By how much, relatively, is the total variation in number of minutes spent on a call-reduced when the number of copiers serviced is introduced into the analysis? Is this a relatively small or large reduction? What is the name of this measure?}

        \paragraph{}
        [TODO ]

      \subsubsection{Calculate $r$ and attach the appropriate sign.}

        \paragraph{}
        [TODO ]


  \part{Ejercicios Montgomery:}

    \paragraph{}
    [TODO ]


  \part{Código Fuente}

    \begin{figure}[!h]
      \centering
      \begin{minted}[frame=single,framesep=5pt]{sas}

      \end{minted}
      \caption{\emph{Código SAS:} [TODO ].}
      \label{code:sas_1}
    \end{figure}
  %-----------------------------
  %  Bibliographic references
  %-----------------------------

  \nocite{rano2017}
  \nocite{sas}
  \nocite{neter1996applied}
  \nocite{montgomery2012introduction}

  \bibliographystyle{acm}
  \bibliography{bib}

\end{document}
